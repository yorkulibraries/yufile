\usepackage[T1]{fontenc}

% Specify the font.
\usepackage{fourier} % Adobe Utopia

\usepackage{microtype}
\microtypecontext{spacing=nonfrench}

% For better formatting in the table listing referees
\usepackage{array}

% Page margins.
\usepackage[margin=2cm]{geometry}

\setlength{\headheight}{15pt}

% Say "A", not "Chapter A", by removing the word \chapter.
\renewcommand{\chaptername}{}

% Candidate's notes
\newcommand*{\notefont}{\fontfamily{phv}\selectfont}

% To include images
\usepackage{graphicx}

% Tighter lists.
\usepackage{enumitem}
\setlist{noitemsep}

% For rules in tables
\usepackage{booktabs}

% For better formatting in the table listing referees
\usepackage{array}

% To allow H location for image floating (meaning "right here")
\usepackage{float}

% Figures float, but minipages don't, so these let me
% me put graphics where I want, labelled, with captions.
% \usepackage{wrapfig}
% \usepackage{caption}

% To include other PDFs. Section and page numbering are all handled perfectly.
\usepackage{pdfpages}
% When including slides do this to make a 2x3 layout:
% \includepdf[pages=-,nup=2x3,delta=5mm 5mm,landscape=false]{filename.pdf}

% I was getting errors like this:
% pdfTeX warning: pdflatex (file ./2-4-1-lrts-frbr-book-review.pdf): PDF inclusion: found PDF version <1.7>, but at most version <1.5> allowed
% but this fixes it:
\pdfoptionpdfminorversion=7

% Suppress warnings like this:
% pdfTeX warning: pdflatex (file ./pdfs/B 2.2.1 Introduction to Data Visualization for Research.pdf): PDF inclusion: multiple pdfs with page group included in a single page
\pdfsuppresswarningpagegroup=1

% For better table formatting
%\usepackage{multirow}
%\usepackage{rotating}
%\usepackage{longtable}

% Start numbering sections at 0.  Because that's where numbering starts.
% \setcounter{section}{-1}
% \setcounter{chapter}{-1}

% Nice headers.
\usepackage{fancyhdr}
\pagestyle{fancy}
\fancyhf{}
% \thepage, \thesubsection etc.  \leftmark is chapter title, \rightmark is section title
% \renewcommand{\headrulewidth}{1pt}

\fancyhead[C]{\nouppercase{\rightmark}}
% \fancyfoot[L]{Name Here}
\fancyfoot[C]{\leftmark}
\fancyfoot[R]{\thepage}

\fancypagestyle{plain}{%
  \fancyhf{}%
  % \fancyfoot[L]{Name Here}
  \fancyfoot[C]{\leftmark}
  \fancyfoot[R]{\thepage}
}

\newcommand{\chaptertitle}{}
\renewcommand{\chaptermark}[1]{\renewcommand{\chaptertitle}{Chapter \thechapter\ #1}}
\renewcommand{\sectionmark}[1]{\markboth{\thesection\ #1}{}}
\renewcommand{\subsectionmark}[1]{\markright{\thesubsection\ #1}}

% Colours red the line above the footer.
\renewcommand{\footrule}{\hbox to\headwidth{\color{red}\leaders\hrule height \headrulewidth\hfill}}

% Turn \url and \href into hyperlinks in PDFs, and pass hyphens option to url package
% (which hyperref calls) to get better line breaks.
% http://tex.stackexchange.com/questions/3033/forcing-linebreaks-in-url?rq=1
\PassOptionsToPackage{hyphens}{url}\usepackage[pdfborder={0 0 0},colorlinks=true,urlcolor=blue]{hyperref}

% With this I can say \link{https://example.com/} and it makes it a hyperlink wrapped in < >
\newcommand{\link}[1]{{\small $<$\url{#1}$>$}}

% Add PDF properties (part of hyperref)
\hypersetup{%
  bookmarksnumbered, % To get A 1.2 etc. into the bookmarks
  pdfauthor={File Preparation Committee},
  pdfsubject={},
  pdftitle={Letters of Reference},
  pdfkeywords={},
  linkcolor=black % Other TOC listings, and internal links, are in red
}

%% Attachments
% \attached puts a short horizontal rule, with some space above and below,
% which I'll follow with a list of the attachments for the section.
% This isn't used, but I'll leave it in just in case.
\newcommand{\attached}{%
\vspace{1em}
\noindent\rule{8cm}{0.4pt}
\newline
\vspace{1em} Attached:
}

%% Candidate's notes
% \candidatenote adds a note from the candidate, formatted (with a
% different typeface) so it looks noticeably different from the rest
% of the file.  The phv font is a Helvetica clone.
\newcommand{\candidatenote}[1] {%
\par\noindent\rule{\textwidth}{1pt}
{\large Candidate's note:}
\begin{quote}
  \fontfamily{phv}\selectfont
  {\small #1}
\end{quote}
}

%% Committee notes
% \committeenote adds a note from the FPC, in italics.
\newcommand{\committeenote}[1] {%
\par\noindent\rule{\textwidth}{1pt}
{\large File Preparation Committee note:}
\begin{quote}
  \itshape{}
  {\small #1}
\end{quote}
}

% No paragraph indent.
% \setlength{\parindent}{0pt}
% \setlength{\parskip}{\baselineskip}

% Remove the page numbers from the TOC.
% \let\Contentsline\contentsline
% \renewcommand\contentsline[3]{\Contentsline{#1}{#2}{}}
